\documentclass[10pt,aspectratio=43,serif]{beamer}		
%\documentclass[handout,t]{beamer}

\batchmode
% \usepackage{pgfpages}
% \pgfpagesuselayout{4 on 1}[letterpaper,landscape,border shrink=5mm]

\usepackage{amsmath,amssymb,enumerate,epsfig,bbm,calc,color,ifthen,capt-of,multimedia,hyperref,ctex,listings}

\usetheme{Berlin}

\DefineNamedColor{named}{socodingblue}    {rgb}{0.296875,0.421875,0.69921875}

\mode<presentation>
\setbeamercolor*{palette primary}{bg=socodingblue,fg=white}
\setbeamercolor*{palette secondary}{bg=socodingblue,fg=white}
\setbeamercolor*{palette tertiary}{bg=socodingblue,fg=white}
\setbeamercolor*{palette quaternary}{bg=socodingblue,fg=white}
\setbeamercolor*{structure}{fg=socodingblue,bg=white}
\setbeamercolor{frametitle}{bg=socodingblue,fg=white}
\mode<all>

\setbeamertemplate{headline}
{%
  \begin{beamercolorbox}[colsep=1.5pt]{upper separation line head}
  \end{beamercolorbox}
  \begin{beamercolorbox}{section in head/foot}
    \vskip2pt\insertnavigation{\paperwidth}\vskip2pt
  \end{beamercolorbox}%
  \begin{beamercolorbox}[colsep=1.5pt]{lower separation line head}
  \end{beamercolorbox}
}

\setbeamertemplate{footline}[frame number]{}

%\setbeamertemplate{navigation symbols}{}

\setbeamertemplate{footline}{}

\catcode`\。=\active
\newcommand{。}{.}

\title{倍增}
\subtitle{区间最值问题、最近公共祖先}
\author{Jialin Rong}
\institute{South China Normal University}
\date{\today}

\AtBeginSection[]
{
  \begin{frame}<beamer>
    \frametitle{目录}
    \tableofcontents[currentsection]
  \end{frame}
}
\beamerdefaultoverlayspecification{<+->}

\begin{document}
% -----------------------------------------------------------------------------

\frame{\titlepage}

%\section[目录]{}
\begin{frame}{目录}
  \tableofcontents
\end{frame}

% -----------------------------------------------------------------------------

\begin{frame}{倍增}

倍增,读音 bèi zēng,汉语词语,意思为成倍增加。

~\\

解释:

\begin{itemize}
  \item 成倍增加。
  \item 变为两倍大。
  \item 增加一倍以上。
\end{itemize}

~\\

\begin{description}
  \item[使线性的处理转化为对数级的处理,大大地优化时间复杂度。] 
\end{description}

\end{frame}

% -----------------------------------------------------------------------------

\section{区间最值问题}
\begin{frame}{引入}

$$a=[1,1,4,5,1,4]$$

\begin{itemize}
  \item $\max_{i=2}^{2}a_i$
  \item $\max_{i=2}^{5}a_i$
  \item $\max(\max_{i=2}^{3}a_i,\max_{i=4}^{5}a_i)$
  \item $\max(\max_{i=2}^{4}a_i,\max_{i=3}^{5}a_i)$
\end{itemize}

\begin{block}{可重复贡献}
$\max(x,x)=x$。即使用来求解的预处理区间有重叠部分,只要这些区间的并是所求的区间,最终计算出的答案就是正确的。
\end{block}

\end{frame}

% -----------------------------------------------------------------------------

\begin{frame}{预处理}

\begin{itemize}
  \item 预处理所有长度为 $2^x \ (x=0,1,2,3,...)$ 的连续区间。
  \item $dp_{i,j}$ 表示从 $i$ 开始 $2^j$ 个数的最大值。
  \item 显然 $dp_{i,0}=a_i$。
\end{itemize}

\begin{description}
 \item \begin{block}{状态转移方程}
$$dp_{i,j}=\max(dp_{i,j-1},dp_{i+2^{j-1},j-1})$$ 
\end{block}
\end{description}


\end{frame}

% -----------------------------------------------------------------------------

\begin{frame}{查询}

求 $\max_{i=L}^{R}a_i$ 的值。

~\\

\begin{itemize}
  \item 用预处理过的区间覆盖 $[L,R]$。
  \item $[L,L+2^s-1],[R-2^s+1,R]$.
  \item $s=\left \lfloor\log_2(R-L+1)\right \rfloor$.
\end{itemize}

\begin{block}{为什么?}
  \begin{itemize}
    \item $2^{\left \lfloor\log_2(R-L+1)\right \rfloor}$ 到底有多长?
    \item 如果是它的两倍,$2^{\left \lfloor\log_2(R-L+1)\right \rfloor + 1}$呢?
  \end{itemize}
\end{block}

\end{frame}

% -----------------------------------------------------------------------------

\begin{frame}{代码实现}

关于实现的细节:

~\\

\begin{itemize}
  \item $\lfloor \log_{2}i \rfloor = \lfloor  \log_{2} \lfloor \frac{i}{2} \rfloor \rfloor +1$
  \item $f_{i,j}$ 初始化?
\end{itemize}

\end{frame}

% -----------------------------------------------------------------------------

\begin{frame}{其他应用}

\begin{itemize}
  \item 区间按位与:$x \ \& \ x = x$
  \item 区间按位或:$x \ | \ x = x$
  \item 区间 GCD:$gcd(x,x)=x$
\end{itemize}

\end{frame}

% -----------------------------------------------------------------------------

\section{最近公共祖先}
\begin{frame}{基本概念}

\begin{itemize}
  \item \textbf{树}:\textit{$n$ 个结点,$n-1$ 条边的连通无向图,任意两个结点之间有且仅有一条简单路径。}
  \item \textbf{深度}:\textit{到根结点的路径上的边数(怎么求?)。}
  \item \textbf{父亲}:\textit{对于除根以外的每个结点,定义为从该结点到根路径上的第二个结点。根结点没有父结点。}
  \item \textbf{祖先}:\textit{一个结点到根结点的路径上,除了它本身外的结点。}
\end{itemize}

\begin{block}{包括它本身?}
\begin{itemize}
  \item 算法导论第十二章 12.2-6:每个结点都是它自己的祖先。
\end{itemize} 
\end{block}

\begin{itemize}
  \item \begin{description}
  \item[\textbf{两个节点的 LCA}:\textit{两个点的公共祖先里面,离根最远的那个。}] 
\end{description}

\end{itemize}

\end{frame}

% -----------------------------------------------------------------------------

\begin{frame}{朴素算法}

在两个点相遇之前...

~\\

\begin{itemize}
  \item 找深度比较大的点。
  \item 往上跳一步。
\end{itemize}

\end{frame}

% -----------------------------------------------------------------------------

\begin{frame}{倍增算法}

\begin{itemize}
  \item 怎么利用倍增性质?
  \item 怎么确保两个点相遇?
\end{itemize}

\end{frame}

% -----------------------------------------------------------------------------

\begin{frame}{预处理}

\begin{itemize}
  \item 节点的深度:DFS.
  \item $fa_{x,y}$ 表示节点 $x$ 往上跳 $2^y$ 步到达的节点。
\end{itemize}

\begin{block}{先跳 $2^{y}$ 步再跳 $2^{y}$ 步...}
$$fa_{x,y} = fa_{fa_{x,y-1},y-1}$$
\end{block}

\begin{itemize}
  \item 只要是没有跳出这颗树就一直预处理。
\end{itemize}

\end{frame}

% -----------------------------------------------------------------------------

\begin{frame}{查询}
\begin{itemize}
  \item 俩节点的深度相差多少?
  \item 怎么利用预处理的数据快速跳?
  \item 十进制转二进制有什么作用?
  \item 不知道跳到哪里怎么确定目标?
  \item 怎么个试探法?
  \item 为什么肯定会试探出来?
\end{itemize}
\end{frame}

% -----------------------------------------------------------------------------

\end{document}
