\documentclass[10pt,aspectratio=43,serif]{beamer}		
%\documentclass[handout,t]{beamer}

\batchmode
% \usepackage{pgfpages}
% \pgfpagesuselayout{4 on 1}[letterpaper,landscape,border shrink=5mm]

\usepackage{amsmath,amssymb,enumerate,epsfig,bbm,calc,color,ifthen,capt-of,multimedia,hyperref,ctex,listings}

\usetheme{Berlin}

\DefineNamedColor{named}{socodingblue}    {rgb}{0.296875,0.421875,0.69921875}

\mode<presentation>
\setbeamercolor*{palette primary}{bg=socodingblue,fg=white}
\setbeamercolor*{palette secondary}{bg=socodingblue,fg=white}
\setbeamercolor*{palette tertiary}{bg=socodingblue,fg=white}
\setbeamercolor*{palette quaternary}{bg=socodingblue,fg=white}
\setbeamercolor*{structure}{fg=socodingblue,bg=white}
\setbeamercolor{frametitle}{bg=socodingblue,fg=white}
\mode<all>

\setbeamertemplate{headline}
{%
	\begin{beamercolorbox}[colsep=1.5pt]{upper separation line head}
	\end{beamercolorbox}
	\begin{beamercolorbox}{section in head/foot}
		\vskip2pt\insertnavigation{\paperwidth}\vskip2pt
	\end{beamercolorbox}%
	\begin{beamercolorbox}[colsep=1.5pt]{lower separation line head}
	\end{beamercolorbox}
}

\setbeamertemplate{footline}[frame number]{}

%\setbeamertemplate{navigation symbols}{}

\setbeamertemplate{footline}{}

\catcode`\。=\active
\newcommand{。}{.}

\title{数论只会 GCD}
\subtitle{“这个我也不会证,要么百度要么背结论吧”}

\author{Jialin Rong}
\institute{South China Normal University}
\date{\today}

\AtBeginSection[]
{
	\begin{frame}<beamer>
		\frametitle{目录}
		\tableofcontents[currentsection]
	\end{frame}
}
\beamerdefaultoverlayspecification{<+->}

\begin{document}

% -----------------------------------------------------------------------------
	
\frame{\titlepage}
	
%\section[目录]{}
\begin{frame}{目录}	
	\tableofcontents
\end{frame}

\begin{frame}{声明}	
	本演示文档可能充斥着各种不严谨甚至是错误的表述,课后请务必参考各种网上资料重新自学一遍有关内容(捂脸)。
\end{frame}
	
% -----------------------------------------------------------------------------

\section{基础知识}
\begin{frame}{整除}
	
	存在整数 $q$ 使得 $b=aq$。
		
	\begin{itemize}
		\item $a \mid b$:$a$ 整除 $b$
		\item $a \nmid b$:$a$ 不整除 $b$
	\end{itemize}

\end{frame}

% -----------------------------------------------------------------------------

\begin{frame}{带余除法}
	
	给定 $a,b$,一定存在唯一一对 $q,r$(都是整数哈)
	
	\begin{itemize}
		\item $a=bq+r$
		\item $0 \leq r < |b|$
	\end{itemize}
	
\end{frame}

% -----------------------------------------------------------------------------

\begin{frame}{整除的性质}
	
	都是整数...
	
	\begin{itemize}
		\item $a\mid b$ 且 $b\mid c \Longleftrightarrow a\mid c$
		\item $a\mid b$ 且 $a \mid c  \Longleftrightarrow$ 对任意 $x,y$ 有 $a\mid bx + cy$
		\item $m \neq 0$,$a\mid b \Longleftrightarrow ma \mid mb$
		\item $a\mid b$ 且 $b \mid a \Longrightarrow a=\pm b$ 
		\item $b\neq 0$ 且 $a \mid b \Longrightarrow |a| \leq |b|$ 
	\end{itemize}	
		
\end{frame}
	

% -----------------------------------------------------------------------------

\begin{frame}{模的性质}
	
	\begin{itemize}
		\item $a \equiv a \pmod m$
		\item $a \equiv b \pmod m\Longleftrightarrow b \equiv a \pmod m$
		\item $a \equiv b \pmod m,b \equiv c \pmod m \Longrightarrow a \equiv c \pmod m$
		\item $a \equiv b \pmod m,c \equiv d \pmod m \Longrightarrow a\pm c \equiv b\pm d \pmod m$
		\item $a \equiv b \pmod m,c \equiv d \pmod m \Longrightarrow ac \equiv bd \pmod m$
	\end{itemize}
	
\end{frame}

% -----------------------------------------------------------------------------

\begin{frame}{整除的性质}
	
	很熟悉的...
	
	\begin{itemize}
		\item $(a+b) \bmod p = (a\bmod p+b\bmod p)\bmod p$
		\item $(a \times b) \bmod p = (a \bmod p )( b \bmod p)\bmod p$
	\end{itemize}
	
\end{frame}

% -----------------------------------------------------------------------------

\section{快速幂}
\begin{frame}{快速幂}
	
	$$a^{m+n}=a^m \times a^n$$
	
	\begin{itemize}
		\item 对指数进行分治
		\item 递归写法 - 两两一组
		\item 非递归写法 - 按需取用
	\end{itemize}
	
\end{frame}

% -----------------------------------------------------------------------------

\section{质数判断}
\begin{frame}{质数判断}
	
	\begin{itemize}
		\item $O(n)$
		\item $O(\sqrt{n})$
		\item $O(\frac{\sqrt{n}}{6})$ - $6k+1,6k+5$
		\item $O(\log n)$ - Miller Rabin 算法
	\end{itemize}
	
\end{frame}

% -----------------------------------------------------------------------------


\end{document}
